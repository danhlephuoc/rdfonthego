
%%%%%%%%%%%%%%%%%%%%%%% file typeinst.tex %%%%%%%%%%%%%%%%%%%%%%%%%
%
% This is the LaTeX source for the instructions to authors using
% the LaTeX document class 'llncs.cls' for contributions to
% the Lecture Notes in Computer Sciences series.
% http://www.springer.com/lncs       Springer Heidelberg 2006/05/04
%
% It may be used as a template for your own input - copy it
% to a new file with a new name and use it as the basis
% for your article.
%
% NB: the document class 'llncs' has its own and detailed documentation, see
% ftp://ftp.springer.de/data/pubftp/pub/tex/latex/llncs/latex2e/llncsdoc.pdf
%
%%%%%%%%%%%%%%%%%%%%%%%%%%%%%%%%%%%%%%%%%%%%%%%%%%%%%%%%%%%%%%%%%%%


\documentclass[runningheads,a4paper]{llncs}

\usepackage{amssymb}
\setcounter{tocdepth}{3}
\usepackage{graphicx}

\usepackage{url}
\urldef{\mailsa}\path|{alfred.hofmann, ursula.barth, ingrid.haas, frank.holzwarth,|
\urldef{\mailsb}\path|anna.kramer, leonie.kunz, christine.reiss, nicole.sator,|
\urldef{\mailsc}\path|erika.siebert-cole, peter.strasser, lncs}@springer.com|    
\newcommand{\keywords}[1]{\par\addvspace\baselineskip
\noindent\keywordname\enspace\ignorespaces#1}

\begin{document}

\mainmatter  % start of an individual contribution

% first the title is needed
\title{Linked Personal Personal Information Management  for Mobile Users}

% a short form should be given in case it is too long for the running head
\titlerunning{Linked Personal Personal Information Management}

% the name(s) of the author(s) follow(s) next
%
% NB: Chinese authors should write their first names(s) in front of
% their surnames. This ensures that the names appear correctly in
% the running heads and the author index.
%
\author{Danh Le Phuoc
\and Anh Le Tuan\and Hanh Hoang Huu}
%
\authorrunning{Danh Le Phuoc et al.}
% (feature abused for this document to repeat the title also on left hand pages)

% the affiliations are given next; don't give your e-mail address
% unless you accept that it will be published
\institute{Digital Enterprise Research Institute ,\\
National University of Ireland Galway, Ireland \\
%\url{http://www.springer.com/lncs}
}
% (feature abused for this document to repeat the title also on left hand pages)

% the affiliations are given next; don't give your e-mail address
% unless you accept that it will be published
%
% NB: a more complex sample for affiliations and the mapping to the
% corresponding authors can be found in the file "llncs.dem"
% (search for the string "\mainmatter" where a contribution starts).
% "llncs.dem" accompanies the document class "llncs.cls".
%

\toctitle{Lecture Notes in Computer Science}
\tocauthor{Authors' Instructions}
\maketitle


\begin{abstract}
The abstract should summarize the contents of the paper and should
contain at least 70 and at most 150 words. It should be written using the
\emph{abstract} environment.
\keywords{mobile, personal information system}
\end{abstract}


\section{Introduction}
[Intro here]

\subsection{Motivated Use cases}

-Semantic Life

-Social network application

-Semantic desktop

\section{Related work}

\section{Data integration and alignment}

[Danh to write this section]

Person-centralized data, data is not only about one personal but about all the social relationship that a person have.

- What kind of data to be taken in paper's consideration

	+ Give examples from data from Semantic Life, Social network application~\cite{Tramp:2010}, Semantic desktops

Data circle : User and his community created data in data silos how can these information can be linked and shared? 
In order to link these information there must be place holders that can tight thing to gather, in this case, person is a centric information,
there fore person identification will be the first class citizen in the personal graphs of data/ personal view of the data/ personal data space. 

As the data comes from different data silos, for example Facebook, twitter, google email, linked in, foaf file, etc, there are different ways to identify 
a personal because there would be several ID(URI, blank node) to identification a person). This fact leads to the  
ID consolidation requirements  Idmesh~\cite{Cudre-Mauroux:2009}, sameAs, ...

- Explain how the ID consolidation can be handled here : Master ID schema


As personal information are accessible as several fragments as graphs with different identification schema. On top of that, the linkages among 
URI/blank nodes are published or transformed in different vocabulary. In order to align these linked information in a same graph to give user a single
view about their personal data space, we need a mechanism for aligning/patching graphs. 

-Explain how to align distributed data into single graph with : mapping rules, multi-layers graph.
 

\section{Context-based information navigation}

[Mr Hanh to outline and  this section]
 
-Context-based information filtering and suggestion

-Context-based query on top of SPARQL and indexing system of mobile triple storage....

\section{System design and implementation}

-Architecture : Wrappers(Facebook, twitter, Phone, linked in, etc) , data storage(phone triple storage), 
ID-reasoner ( SameAs reasoner), content provider ( [Tuan Anh to draw the architecture and describe 
component here])

-System design

-Implementation details
	
	+triple storage for mobile
	
	+ Wrapper
	
	+Query interfaces 

\section{Demo and evaluation}

-Describe demo and screen shot

-Some evaluations: Evaluation metrics : query response time vs size of data?????

\section{Conclusion and future work}

%
% The following two commands are all you need in the
% initial runs of your .tex file to
% produce the bibliography for the citations in your paper.
\bibliographystyle{abbrv}
\bibliography{spim2u}
\end{document}
